\documentclass{article}

% ----- ОБРАБОТКА РУССКОГО ЯЗЫКА -----
\usepackage{polyglossia}
\setmainlanguage{russian}
\newfontfamily\cyrillicfont{Roboto}[Script=Cyrillic]


% ----- ГЕОМЕТРИЯ ДОКУМЕНТА -----
\usepackage{geometry}
\geometry{
	a4paper,
	textwidth=160mm,
	top=20mm,
	bottom=25mm
}


% ----- ШРИФТЫ -----
\usepackage{fontspec}
\usepackage{fontawesome}
\usepackage[default, opentype]{sourcesanspro}
\defaultfontfeatures{Ligatures=TeX}

\newfontfamily\headerfont[
	Path=fonts/,
	UprightFont=*-Regular,
	ItalicFont=*-Italic,
	BoldFont=*-Bold,
	BoldItalicFont=*-BoldItalic
]{Roboto}
\newcommand*{\bodyfont}{\sourcesanspro}


% ----- СТИЛИ ЗАГОЛОВКОВ И ТЕКСТА -----
\usepackage{mfirstuc}
\usepackage{xcolor}

\definecolor{awesome}{HTML}{87A96B}

\newcommand*{\headerstyle}[1]{\fontsize{16pt}{1em}\headerfont #1}
\newcommand*{\vacancystyle}[1]{{\fontsize{24pt}{1em}\bfseries\headerfont\color{awesome} #1}}
\newcommand*{\parstyle}[1]{\fontsize{9pt}{1em}\bodyfont #1}


% ----- ИЗОБРАЖЕНИЯ -----
\usepackage{graphicx}
\graphicspath{{report_files/}}


% ----- НОВЫЕ ВСПОМОГАТЕЛЬНЫЕ КОМАНДЫ -----
\newcommand{\contSep}{\quad\textbar\quad}
\input{report_files/variables.tex}


% ----- КОЛОНТИТУЛЫ -----
\usepackage{fancyhdr}
\renewcommand{\headrulewidth}{0pt}
\fancyhf{}
\fancyfoot[C]{\footnotesize\thepage}


% ----- ГИПЕРССЫЛКИ -----
\usepackage[hidelinks]{hyperref}
\hypersetup{colorlinks=false}




\begin{document}
\pagestyle{fancy}
\headerstyle{Анализ рынка вакансий по запросу:}
\vspace{5pt}

\vacancystyle{\capitalisewords{\VacancyName}}

\parstyle{\faGithub\quad \href{https://github.com/sepremento/}{sepremento}
\contSep\faTelegram\quad \href{https://t.me/sepremento}{sepremento}}
\vfill

Всего вакансий по этому запросу: \enskip\TotalVac \contSep Обработано вакансий
по этому запросу:\enskip \ParsedVac
\vspace{20pt}

\includegraphics[width=\textwidth, height=0.4\textheight]{freq_tags.png}

\noindent Средняя зарплата по этой вакансии: \enskip\MeanSalary\enskip руб. \\
Медианная зарплата по этой вакансии: \enskip\MedianSalary\enskip руб.
\vfill

\includegraphics[width=\textwidth, height=0.4\textheight]{salaries.png}

\newpage
Список ключевых слов - результат лингвистического анализа. Сначала
рассматриваются слова из тэгов вакансий, для каждого из них подбирается по сто
максимально похожих слов. Формируется общий список и из этого списка выбирается
сорок наиболее релевантных слов на английском языке, что даёт понимание о стэке
технологий данной вакансии. 
\vfill
\includegraphics[width=\textwidth]{keywords.png}

\includegraphics[width=\textwidth]{geography.png}
\vfill

\includegraphics[width=\textwidth]{dates.png}

\end{document}
